% 和文概要
\begin{jabstract} 
プロジェクトは、ビーコンIoT(以下、ビーコン)を函館の街の様々な場所に設置し、新たな体験や魅力を提供することを目的とする。

函館市は、「食」「歴史」「景色」など、様々な面で魅力があふれている。
食という点では、函館市は海に面していることで、新鮮でおいしい海産物を食べることができる。
また、歴史に残る出来事が数多くあったため、歴史的建造物も多数存在している。
さらに、「函館の夜景」は日本3大夜景の1つである。
このような魅力的な観光資源が多く存在しているによって、観光客が毎年多く訪れる。
令和元年度における観光客入込客数は、合計約536万9千人であった。
このことから、観光産業は、函館市にとって重要な基盤産業となっていることがわかる。
そのため、何度も訪れてもらえるよう、様々な角度から魅力を発信し続ける必要があると考えられる。
近年、通信技術の向上や通信機器の小型化・低コスト化のためIoTが普及している。
私たちは、ビーコンを用いることによって、函館の街の様々な魅力を、新しい方法で、より多くの人に提供したいと考えている。
その目的を実現するために、活動を行う上でいくつかの工夫を凝らしている。
1つ目は、アジャイル開発手法をもとにした独自のタスク管理手法を用いていることである。
バックログを導入することにより、タスク管理が容易となった。
また、開発フェーズに移行した際も、アジャイル開発手法を導入しやすいという利点がある。
2つ目は、プロジェクトリーダーを決めていないことである。
これにより、メンバー全員が責任感を持ち、活動しやすい環境が実現されている。

サービスの提案をするにあたり、フィールドワークを実施し、街に潜む課題・需要を見つけ出した。
さらに、「函館らしさ」「ビーコンである理由」「必要性」「新規性」「不変的な魅力(継続性)」の観点から、複数のアイデアを創出し、9個のアイデアに絞り込んだ。
後期では、まず、これらのアイデアをさらにブラッシュアップし、3つまたは4つのアイデアに絞り込む。
その後、開発を行う際に必要となるサービスの詳細設計や、要件定義の作成を行う。
それらをもとに、サービスの実装と実証実験を行う。
このプロセスを繰り返し行うことで、よりよいサービスを作る。

% 和文キーワード
\begin{jkeyword}
ビーコンIoT, 函館, 観光資源, 基盤産業, タスク管理手法, 街に潜む課題・需要
\end{jkeyword}
\bunseki{未来太郎}
\end{jabstract}

%英語の概要
\begin{eabstract} Abstract in English. 


% 英文キーワード
\begin{ekeyword}
Keyrods1, Keyword2, Keyword3, Keyword4, Keyword5
\end{ekeyword}
\bunseki{久末瑠紅}
\end{eabstract}

