% 和文概要
\begin{jabstract} 
本プロジェクトは、ビーコンIoT(以下、ビーコン)を函館の街の様々な場所に設置し、新たな体験や魅力を提供することを目的とする。

函館市は、「食」「歴史」「景色」など、様々な面で魅力があふれている。
食という点では、函館市は海に面していることで、新鮮でおいしい海産物を食べることができる。
また、歴史に残る出来事が数多くあったため、歴史的建造物も多数存在している。
さらに、「函館の夜景」は日本3大夜景の1つである。
このような魅力的な観光資源が多く存在していることによって、観光客が毎年多く訪れる。
令和元年度における観光客入込客数は、合計約536万9千人であった。
このことから、観光産業は、函館市にとって重要な基盤産業となっていることがわかる。
そのため、何度も訪れてもらえるよう、様々な角度から魅力を発信し続ける必要があると考えられる。
近年、通信技術の向上や通信機器の小型化・低コスト化のためIoTが普及している。
私たちは、ビーコンを用いることによって、函館の街の様々な魅力を、新しい方法で、より多くの人に提供したいと考えている。
その目的を実現するために、活動を行う上でいくつかの工夫を凝らしている。
1つ目は、アジャイル開発手法をもとにした独自のタスク管理手法を用いていることである。
バックログを導入することにより、タスク管理が容易となった。
また、開発フェーズに移行した際も、アジャイル開発手法を導入しやすいという利点がある。
2つ目は、プロジェクトリーダーを決めていないことである。
これにより、メンバー全員が責任感を持ち、活動しやすい環境が実現されている。

サービスの提案をするにあたり、フィールドワークを実施し、街に潜む課題・需要を見つけ出した。
さらに、「サービスの函館らしさ」「ビーコンである理由」「サービスの必要性」「サービスの新規性」「サービスの不変的な魅力(継続性)」の観点から、複数のアイデアを創出して9個に絞り込んだ。
後期では、これらのアイデアをさらにブラッシュアップし、3つまたは4つのアイデアに絞り込む。
その後、開発を行う際に必要となるサービスの詳細設計や、要件定義の作成を行う。
それらをもとに、サービスの実装と実証実験を行う。
このプロセスを繰り返し行うことで、よりよいサービスを作る。

% 和文キーワード
\begin{jkeyword}
ビーコンIoT, 函館, 観光資源, 基盤産業, タスク管理手法, 街に潜む課題・需要
\end{jkeyword}
\bunseki{久末瑠紅}
\end{jabstract}

%英語の概要
\begin{eabstract} Abstract in English. 
The purpose of this project is install beacons in Hakodate real downtown, and develop services that utilize beacons. Through the service, create new value and provide people unique experiences and attractions.

Hakodate City has lots of attractions such as "food," "history," and "scenery."
In terms of food, Hakodate City faces the sea, so you can eat fresh and delicious seafood.
In addition, there are many historical buildings because many historical events was happened in Hakodate city.
Furthermore, "Night view of Hakodate" is one of the top 3 night views in Japan.
Due to the existence of many such attractive tourism resources, many tourists visit every year.
The total number of tourists was about 5,369,000, in 2019.
From this number, it can be seen that the tourism industry has become an important basic industry for Hakodate City.
Therefore, it is necessary to continue to tell the attractions from various angles so that Hakodate city make people think they wants to visit this attractions city many times.
These days, IoT has become widespread due to improvements in communication technology, miniaturization and cost reduction of communication equipment.
We would like to use beacons to bring the attractions to more people in new ways.
In order to achieve that purpose, we devised our activities.
First, we use a original task management method based on agile development methods.
By introducing the backlog, managing tasks became easier.
In addition, using task management method makes it is easy to introduce agile development methods when we move to the development phase.
The second is that we didn't decide a project leader.
As a result, an environment is realized in which all members have responsibility and are easy to work.

We did fieldwork for proposing services, and identified issues and demands hidden in Hakodate city.
In addition, we created some ideas from the perspectives of "how typical of hakdoate," "why we use beacon," "necessity of service," "novelty of service," and "unchangeable attractive (continuity) of service."
We narrowed it down nine ideas using that view points.
In the second semester, we will brush-up and further refine these ideas to three or four ideas.
After that, we are going to make statement and  requirement definition of the services for developing.
Based on them, we will make implement services and conduct demonstration experiments.
By repeating this process, we improve the quality of the services. 
% 英文キーワード
\begin{ekeyword}
Beacon IoT, Hakodate, tourism resources, basic industry, task management method, issues and demand hidden in Hakodate city
\end{ekeyword}
\bunseki{Ryuku Hisasue}
\end{eabstract}
