\chapter{提供するサービスについて}
2.2節で具体化した各小課題の解決のプロセスの概要を、各々記述する。
\section{概要}
私たちは、ビーコンを用いて函館という地域に付加価値をもたらす方法として以下の9個のサービスを提案する。サービスは「サービスの函館らしさ」「ビーコンである理由」「サービスの必要性」「サービスの新規性」「サービスの不変的な魅力」の5つの観点に重点をおいて選定した。
\section{サービスについて}
\subsection{俺を食ってくれぇ!}
\begin{enumerate}
    \item 背景
    \par 函館は海に囲まれた街であり、豊富な海産物に恵まれているという特徴がある。朝市やスーパーには、獲れたてのイカやホッケなど、新鮮な海産物が並ぶ。これらの海産物は飲食店だけでなく、地域住民や観光客からも多く求められている。
    \item サービス概要
    \par 朝市やスーパーの海産物にビーコンを取り付け、海産物の近くを通った客に、海産物の鮮度・その海産物の状態に最適な調理方法の情報を発信する。本サービスでは、「海産物が客に自身を美味しく食べて欲しいとアピールしてくる」という新たな体験を提供したいと考えている。
    \item 目的
    \par 地域住民や観光客が函館の新鮮な海産物を買う際に、どの海産物の鮮度が良いのか、どのように調理して良いのか分からない。しかし、店員に話しかけるのは難易度が高い場合もある。本サービスでは、このような課題を解決できると考える。
    \item 5つの観点
    \begin{itemize}
       \item サービスの函館らしさ
        \par 函館は海産物が特産品であるため、函館らしいといえる。
        \item ビーコンである理由
        \par ビーコンの電波の受信範囲内に入ったときにのみ情報を受信するため、ビーコンでなくてはならない。
        \item サービスの必要性
        \par 知識がなくても目利きができるようになるため、誰でも新鮮な食材を見分けることができるようになる。また、鮮度に合わせた調理方法を提案するため、どのような食材でも美味しく調理することができる。
        \item サービスの新規性
        \par 本サービスでは、食材からユーザーにアピールしてくるという、新たな体験をすることができる。
        \item サービスの不変的な魅力
        \par 専門的な知識が不必要になるため、幅広いユーザーに永続的に使用されると考えられる。
    \end{itemize}
\end{enumerate}

\subsection{函館のここ、おすすめかも}
\begin{enumerate}
    \item 背景
    \par 函館には有名な店はもちろん、昔ながらのローカルなお店もたくさんある。観光客はそれら全ての魅力に気づくことができず、函館旅行を満喫できているとはいえない。
    \item サービス概要
    \par ビーコンとGPSを連携させ、屋内外問わず位置情報を取得し、ユーザーの行動履歴に基づいたおすすめの場所の提案をおこなう。
    \item 目的
    \par COVID-19が収束した後の函館を想定したサービス。環境客に函館の街を楽しんでもらうことや、幅広く函館の街を知ってもらうことがこのサービスの目的である。
    \item 5つの観点
    \begin{itemize}
        \item サービスの函館らしさ
        \par 函館には様々な観光名所がある。また、函館には地元ならではのお店もたくさん存在するため、函館らしいといえる。
        \item ビーコンである理由
        \par 屋内のお店に訪れる際、屋内の位置情報はGPSでは取得することが難しいのでビーコンでサービスを実現する必要がある。また、ビーコンを用いることでどの店舗にどれくらい滞在したかなどの情報も取得することができるため、ビーコンでの実装をするべきである。
        \item サービスの必要性
        \par 手軽に、深く函館について知る機会を提供することで、観光客が増え域内循環が向上される。
        \item サービスの新規性
        \par 手軽に、深く函館について知る機会を提供することができる。
        \item サービスの不変的な魅力
        \par 函館は海産物が特産品であるため、函館らしいといえる。
    \end{itemize}
\end{enumerate}

\subsection{目の前のイベントは何?}
\begin{enumerate}
    \item 背景
    \par 函館では展示や物販など様々なイベントがおこなわれている。それらのイベントには、イベントの目的や詳細など、一見では知ることができない情報がある。私たちはそれらの情報に手軽に触れることができるサービスを提案する。
    \item サービス概要
    \par このサービスはイベントの主催者や出店者、イベントに立ち寄った観光客や地元の人を対象としたサービスであり、立ち寄った観光客や地元の人が自身のスマートフォンから手軽にイベントの詳細を知ることがこのサービスの主な機能である。立ち寄った観光客や地元の人が自身のスマートフォンから手軽にイベントの詳細を知ることができる。さらに、イベントに参加した観光客や地元の人によるイベントのレビュー機能を考案している。この機能によって、イベントに参加した観光客や地元の人はイベントや主催者を評価することができる。
    \item 目的
    \par このサービスを提供することでイベントの活性化や発展に役立て、函館のさらなる活性化につなげることを目的としている。    
\begin{itemize}
        \item サービスの函館らしさ
        \par 背景でも記した通り、函館はイベントが多い街である。よってこのサービスは函館らしさを満たしているといえる。
        \item ビーコンである理由
        \par 複数のイベントが同じ場所で同時におこなわれていたとき、GPSでは本当に「目の前の」イベントかどうか判断することができない。また、ビーコンを用いることでイベントの方から観光客や地元の人にイベントについて働きかけることができる。これらの要因からこのサービスではビーコンを用いる必要がある。
        \item サービスの必要性
        \par 目の前で開催されているイベントについて初参加だとわからないことが多いため、イベントを十分に楽しめないことがある。よって、このサービスは必要でる。
        \item サービスの新規性
        \par 前例として「Beaconを利用した観光ガイドアプリ」[1]が存在するがこのサービスではイベントの内容だけでなく、イベントの主催者についてや目的なども知ることができる。またイベントをレビューする機能はイベントの主催者と参加者が手軽に交流する機会として新規性があるといえる。
        \item サービスの不変的な魅力
        \par このサービスはイベントがおこなわれている限り使うことができるため、不変的な魅力があるといえる。
    \end{itemize}
\end{enumerate}

\subsection{函館を舞台としたADV}
\begin{enumerate}
    \item 背景
    \par 函館はその街並みが美しいことから、観光地としてだけでなく、映画やCMのロケ地としてもよく使用される。しかしながら、函館市ではロケ地を(あんまり推してないよねって書きたい)。
    \item サービス概要
    \par 函館を舞台としたゲームを作成し、物語のキーとなる地点にビーコンを設置する。ユーザーはその地点を訪れることで、そこを舞台とした限定ストーリーやスチルなどの特典を獲得することができる。
    \item 目的
    \par 函館を舞台としたゲームを作ることで、函館の街並みを市外のユーザーにも伝えることができる。また、ゲームを楽しんでもらうことで、実際にゲームの舞台となった場所を訪れてもらうことができると考える。
    \item 5つの観点
    \begin{itemize}
        \item サービスの函館らしさ
        \par 函館を舞台としたゲームを作成するため、函館らしいといえる。
        \item ビーコンである理由
        \par 場所によっては、QRコードを置くことが難しい場合がある。また、確実にその場所を訪れたユーザーのみが特典を得られるようにするためには、ビーコンが最適であると考える。
        \item サービスの必要性
        \par 現在、函館の魅力を伝えるアプリケーションは全くリリースされていない。そのため、函館の魅力を伝えるアプリケーションを作成する必要がある。
        \item サービスの新規性
        \par ただ観光地やロケ地を巡るだけでなく、そこでさらに特典を得ることができる。また、函館市と協力してこのサービスを実現することができれば、函館のロケ地巡りを新たな魅力として推進できる。
        \item サービスの不変的な魅力
        \par 現地を訪れることで得られる特典は、継続的にアップデートすることができる。さらに、函館の新たな魅力として定着させることができると考える。
    \end{itemize}
\end{enumerate}

\subsection{クーポン長者}
\begin{enumerate}
    \item 背景
    \par (・ω・`)
    \item サービス概要
    \par 店の近くを通りかかった人のスマートフォンに別の店のクーポンを提示して店に呼び込む。呼び込みが成功すると別の店のクーポンを得た人は、その別の店に足を運ぶ。この動作を繰り返すことで函館を活発にする。
    \item 目的
    \par 最初のクーポンをもらった店から、他のお店をめぐることで函館全体の活性化につなげる。さらに、わらしべ長者のような体験を店に立ち寄った人に与える。
    \item 5つの観点
    \begin{itemize}
        \item サービスの函館らしさ
        \par 函館は観光地であるため店が多い。よって函館らしさはあるといえる。
        \item ビーコンである理由
        \par 店に滞在していることを正確に把握する必要がある他、店から客にクーポンを提示してもらう必要があるためビーコンが必要である。
        \item サービスの必要性
        \par このサービスを提供することで、客に店のはしごを促すことができる。はしごを促すことで地域全体の活性化につながる。
        \item サービスの新規性
        \par 店が別の店のクーポンを配るサービスは今までに存在しない。また、このサービスは店単体の活性化ではなく、地域全体の活性化につながるサービスであり新規性がある。
        \item サービスの不変的な魅力
        \par このサービスは店があるかぎり存在することができる。
    \end{itemize}
\end{enumerate}

\subsection{未来大のライブラリを良くしよう}
\begin{enumerate}
    \item 背景
    \par 未来大学のライブラリでは、パソコンで蔵書を検索し、パソコンに出力された番号のメモを取る。この番号をもとに本の位置を把握する。このメモを取るというアナログな行為によって検索から本の位置把握までの流れがスムーズではないのが現状である。
    \item サービス概要
    \par 本棚にビーコンを設置し、スマートフォンで電波を受信することで目的の本の位置を把握することができる。送信する情報の中に、本のレビュー機能や講義に適している参考書のおすすめ情報を付加する。
    \item 目的
    \par サービスを使用することで、探している本の位置把握を簡単にする。また、何か本を借りたいという明確でない目的でライブラリを訪れた人に対してレビュー機能等でお勧めの本を提案することができる。
    \item 5つの観点
    \begin{itemize}
        \item サービスの函館らしさ
        \par 未来大学向けのサービスであるため函館らしさは薄いと考える。
        \item ビーコンである理由
        \par GPSは屋内での位置測位の精度が低い。加えてライブラリという小さな範囲となるとGPSの位置把握は困難であると考える。一方、ビーコンを用いた屋内での位置測位は他の屋内でも位置測位と比較すると導入が容易なため、ライブラリへの導入はビーコンが適している。
        \item サービスの必要性
        \par 目的の本がある場合は、本の位置がすぐに把握できるため、以前までの検索結果のメモを取るという手間を省くことができる。特定の分野の参考書などを借りたいなどの場合は、レビュー機能で本の情報を得ることで、分野の中でもどれがおすすめなのかなど、本を選ぶ手助けができる。
        \item サービスの新規性
        \par 大学ならではの特徴として、講義関連の参考書の表示や本のレビュー機能を加えるため、図書館にはないサービスを提供できる。
        \item サービスの不変的な魅力
        \par 大学のライブラリが存続する限り使用されていくと考える。
    \end{itemize}
\end{enumerate}

\subsection{観光客x地元の人}
\begin{enumerate}
    \item 背景
    \par 観光では地元の人に道やおすすめの店を知りたいという場面が存在する。一方地元の人は観光客の人を助けたい、案内したいという場面が存在する。実際は、観光客は近くの人に聞いたり、地元の人は言語がわからないまま対応する場面があり、どちらも心理的負荷が高いように見受けられる。
    \item サービス概要
    \par 地元の人の方はビーコンを持ち、観光客の方はスマートフォンで地元の人を探すことができる。また、ビーコンから受け取る情報は位置だけでなく、対応言語など地元の人の情報を知ることができる。
    \item 目的
    \par このサービスは日本人・外国人問わず、様々な人とコミュニケーションをとることで、観光客が訪れやすい場所になり、観光客を増やすことができる。また、言語に対応した案内をすることで、観光客が快適に函館の街を楽しむことができる。
    \item 5つの観点
    \begin{itemize}
        \item サービスの函館らしさ
        \par 観光客が地元の人しか知らないような情報を得ることができる。
        \item ビーコンである理由
        \par 近くにいる人がどこにいるかを探すためにはGPSではなくビーコンが適している。GPSではサービスの利用者以外にも位置が把握されてしまうため、安全性に欠ける。
        \item サービスの必要性
        \par 観光客は相手の情報を知ることができるため、助けを求めるときの心理的負荷が小さくなる。地元の人は他言語を実際に使用する機会を得ることができる。
        \item サービスの新規性
        \par 地元の人の情報をその場で知る術はなかったが、このサービスを提供することで実現可能になる。
        \item サービスの不変的な魅力
        \par 多くの観光客が訪れる函館では、このような場面が完全になくなることはないと考える。
    \end{itemize}
\end{enumerate}

\subsection{造語シェア}
\begin{enumerate}
    \item 背景
    \par ネット用語など、若者を中心に様々な言葉が生まれる現代、言葉の変遷を函館の魅力的な景観から発信することで新しい函館の歩き方を体験できる。
    \item サービス概要
    \par 観光地や景色の良いところにビーコンを設置し、そのビーコンを受信できる範囲から見えた景色をまだ存在しない新しい言葉で表現し投稿、シェアする。また良いと思った単語に投票することができる。
    \item 目的
    \par 函館の魅力的な景観から造語を作ってシェアするサービスを提供することで、函館の新しい楽しみ方をユーザーに体験してもらうこと。
    \item 5つの観点
    \begin{itemize}
        \item サービスの函館らしさ
        \par 函館の景色を使ったサービスなので函館らしいといえる。
        \item ビーコンである理由
        \par ビーコンの検知する範囲内でのみ投稿できるるので、投稿した人は実際にその場に行って景色を見た人であると言い切ることができる。
        \item サービスの必要性
        \par 行きたい場所を単語で検索し、造語がヒットすることで投稿者の景色から感じたことを共有することができる。
        \item サービスの新規性
        \par 言葉を作るサービスは今までに存在しない。
        \item サービスの不変的な魅力
        \par 見る人によって景色から感じることは違うため、このサービスは同じ景色でも使う人によって異なる単語を作ることができる。
    \end{itemize}
\end{enumerate}

\subsection{ビーコンバトラー}
\begin{enumerate}
    \item 背景
    \par (・ω・`)  
    \item サービス概要
    \par 函館の街中に設置されたビーコンの電波を受信すると、そのビーコンのUUIDに基づいたキャラクターが生成される。ビーコンごとに生成されるキャラクターとステータスが異なるため、より強いキャラクターを探し、対戦するゲームである。
    \item 目的
    \par (・ω・`)
    \item 5つの観点
    \begin{itemize}
        \item サービスの函館らしさ
        \par 函館の街が舞台であり、函館にちなんだキャラクターを登場させる。
        \item ビーコンである理由
        \par ビーコンのUUIDを活用したサービスである。
        \item サービスの必要性
        \par 函館では、朝市や金森赤レンガ倉庫、五稜郭といった観光地にビーコンが設置されている。このビーコンを使用することで、ユーザーに函館のお店を利用してもらうことができると考えられる。
        \item サービスの新規性
        \par ビーコンのUUIDを活用したサービスは現時点で前例がない。
        \item サービスの不変的な魅力
        \par 定期的に生成されるキャラクターを更新するため、継続的にプレイすることができる。また、観光をより楽しむことができるようになると考える。
    \end{itemize}
\end{enumerate}
\bunseki{袴田、野間}

% 参考文献
\begin{thebibliography}{9}
 \bibitem {1} 株式会社リサーチアンドソリューション, Beaconを活用した観光ガイドアプリ~魅力的な街創りをサポート~, 閲覧日 2020-9-17, https://www.rands-co.com/lp/beacon.html
\end{thebibliography/}
