\chapter{学び}

\section{情報をアウトプットする技術}
前期では、ロゴ製作、フィールドワーク、サービス考案というアイデアをアウトプットする活動を行ってきた。
ロゴ製作では、メンバー各自がプロジェクトのイメージを考えロゴをデザインし、製作したロゴのイメージをオンライン発表を通して他メンバーに伝えた。

フィールドワークでは、函館市内5ヶ所にフィールドワークを行った。
フィールドワークを行った場所は、「五稜郭」「赤川・美原・昭和・富岡」「函館駅・西部地区」「七飯・北斗」「空港・湯の川」の5ヶ所である。
この5ヶ所を回り、思ったことや気づいたことについてGoogle Jamboradを使用して、付箋に書き出し、似ている意見にまとめてラベルを付け、KJ法を用いて意見の関連を明らかにした。

サービス考案では、フィールドワークのまとめや自身の経験からサービスのアイデアを考えた。
その際、BSKJ法に加え、時間内にテーマに沿ったアイデアを考案するという手法(ハッカソン形式)を用いて、サービスのアイデアをより良いものにした(洗練させた)。
またSlackワークスペース内に、思いついたアイデアを誰でも自由に書き込めるチャンネルを作ることで、ユニークなアイデアを発見できる機会を増やすことができた。
小規模なOSTを用いて、自身が興味のあるアイデアに対してさらに議論を深めていく方式をとることにより、より活発な議論を行うことができた。

最後にプロジェクトの中間発表では、スライドを用いたプロジェクト概要の説明を動画やポスターで行った。
より多くの人に興味を持ってもらうために、発表動画にイメージキャラクターを起用した。
さらに、発表動画では合成音声を用いることによって、印象に残る動画を目指した。
聞き手に伝わるようにスライドとポスターの文章を添削し、聞き手がわかりにくいと感じた部分を修正することによって、アウトプットの技術を高めることができた。
\bunseki{大石晃平}

\section{サービス設計}
前期のプロジェクトを通して、どのようにしてサービスを設計するかを学ぶことができた。
コロナ禍というのを加味して、今回提案するサービスの「函館らしさ」「ビーコンである理由」「必要性」「新規性」「不変的な魅力」という点についてブレインストーミングやKJ法を用いて徹底的に議論した。
今回提案するサービスと類似した既存サービスとの相違について考えることで、サービスの新規性を深めることができた。
また、教員やTA、他のグループメンバーによるレビューにより、問題点を繰り返し修正した。
これにより、サービスを考案・設計する技術を向上させることができた。
\bunseki{熊谷浩平}

\section{リーダーを設けないチーム運営}
前期の活動では、初めての試みとして、チーム内でリーダーを決めず、進行役を週ごとの交代制にした。
これによって特定の人に負荷が集中しにくくなった。
スケジュールを管理する際には、タスクの内容を優先度順に並び替え、それを全員で共有して作業できた。
その際、全員が対等な立場にあるため、意見が述べやすくなる効果があった。
さらに、全員の意見が一致するまで話し合いを行うことで、意見や意識を一つの方向に定めることができた。

一方で、進行役交代時の情報共有が不足していたため、スケジュールが決まっていても、具体的な作業内容の見通しをつけにくいという課題があった。
さらにスケジュールを決定する話し合いが長くなる傾向があり、本質的な作業にとりかかるのが遅れることがあった。
これらを解決するためには、全員の同意を取るなどの意思決定作業を自動化することにより、より具体的な作業内容の話し合いに時間を使うべきであると考える。
\bunseki{久米田羽月}

\section{振り返り}
\subsection{スケジュール管理}
ファシリテータを中心に、Google Jamboardを使用して直近に行う必要があるタスクのバックログを作成した。
その際、重みの基準となるタスクを設定し、タスクごとの重みを全員で話し合うことで、優先度などを考慮した計画を立てるのに役立てた。
各週を担当するファシリテータはバックログを基に次週行うタスクを選択し、それに従ってプロジェクトの管理を行った。
この作業を行うことで、いつまでに何をして、これからは何をするべきなのかをプロジェクトのメンバー全員で共有することができた。
\bunseki{久米田羽月}

\subsection{情報共有}
ロゴの作成やフィールドワークから得られたこと、アイデア出しでの意見交換、アイデア案の発表など、多くの場面で情報共有を行った。
特に今年度はコロナウイルス感染拡大を考慮し、Zoomを使用したオンラインでの活動が主となった。
そのため、情報共有の手段としてSlackやesa、Google Jamboardなどのオンラインツールを活用し、議事録やブレインストーミングの結果を文章に残すことで、網羅的に情報を共有できるような工夫を行った。

また、プロジェクト活動や自主的な勉強会を動画として残しておくことで、止むを得ず欠席した場合や、振り返りのために役立てることができた。
\bunseki{熊谷浩平}
