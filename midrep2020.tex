% プロジェクト学習中間報告書書式テンプレート ver.1.0 (iso-2022-jp)

% 両面印刷する場合は `openany' を削除する
\documentclass[openany,11pt,papersize]{jsbook}

% 報告書提出用スタイルファイル
%\usepackage[final]{funpro}%最終報告書
\usepackage[middle]{funpro}%中間報告書

% 画像ファイル (EPS, EPDF, PNG) を読み込むために
\usepackage[dvipdfm]{graphicx}

 \def\hissu{\bgroup\color{red}}
 \def\endhissu{\egroup}


% ここから -->
%\usepackage{calc,ifthen}
%\newcounter{hoge}
%\newcommand{\fake}[1]{\whiledo{\thehoge<70}{#1\stepcounter{hoge}}%
%  \setcounter{hoge}{0}}
% <-- ここまで 削除してもよい

% 年度の指定
\thisYear{2020}

% プロジェクト名
\jProjectName{ビーコンIoTで函館のまちをハックする -Beacon FUN 4-}

% [簡易版のプロジェクト名]{正式なプロジェクト名}
% 欧文のプロジェクト名が極端に長い(2行を超える)場合は、短い記述を
% 任意引数として渡す。
%\eProjectName[Making Delicious curry]{How to make delicious curry of Hakodate}
\eProjectName{Leverage the Beacon IoT for Our Smarter Life in Hakodate Real Downtown -Beacon FUN 4-}


% <プロジェクト番号>-<グループ名>
\ProjectNumber{7}

% グループ名
\jGroupName{}
\eGroupName{}

% プロジェクトリーダ      注意: 学籍番号不用
\ProjectLeader{未来花子}{Hanako~Mirai}

% グループリーダ      注意: 学籍番号不用
\GroupLeader  {未来太郎}{Taro~Mirai}

% メンバー数
\SumOfMembers{15}
% グループメンバ      注意: 学籍番号不用
\GroupMember  {1}{久末瑠紅}{Ryuku~Hisause}
\GroupMember  {2}{}{}
\GroupMember  {3}{}{}
\GroupMember  {4}{}{}
\GroupMember  {5}{}{}
\GroupMember  {6}{}{}
\GroupMember  {7}{}{}
\GroupMember  {8}{}{}
\GroupMember  {9}{}{}
\GroupMember  {10}{}{}
\GroupMember  {11}{}{}
\GroupMember  {12}{}{}
\GroupMember  {13}{}{}
\GroupMember  {14}{}{}
\GroupMember  {15}{}{}
% 指導教員
\jadvisor{松原克弥,藤野雄一,鈴木昭二,奥野拓,鈴木恵二}
% 複数人数いる場合はカンマ(,)で区切る。カンマの前後に空白は入れない。
\eadvisor{Katsuya~Matsubara,Yuichi~Fujino,Sho´ji~Suzuki,Taku~Okuno,Keiji~Suzuki}

% 論文提出日
\jdate{2020年9月18日}
\edate{September~18, 2020}

\begin{document}
%
% 表紙
\maketitle

%前付け
\frontmatter

% 和文概要
\begin{jabstract} 
プロジェクトは、ビーコンIoT(以下、ビーコン)を函館の街の様々な場所に設置し、新たな体験や魅力を提供することを目的とする。

函館市は、「食」「歴史」「景色」など、様々な面で魅力があふれている。
食という点では、函館市は海に面していることで、新鮮でおいしい海産物を食べることができる。
また、歴史に残る出来事が数多くあったため、歴史的建造物も多数存在している。
さらに、「函館の夜景」は日本3大夜景の1つである。
このような魅力的な観光資源が多く存在しているによって、観光客が毎年多く訪れる。
令和元年度における観光客入込客数は、合計約536万9千人であった。
このことから、観光産業は、函館市にとって重要な基盤産業となっていることがわかる。
そのため、何度も訪れてもらえるよう、様々な角度から魅力を発信し続ける必要があると考えられる。
近年、通信技術の向上や通信機器の小型化・低コスト化のためIoTが普及している。
私たちは、ビーコンを用いることによって、函館の街の様々な魅力を、新しい方法で、より多くの人に提供したいと考えている。
その目的を実現するために、活動を行う上でいくつかの工夫を凝らしている。
1つ目は、アジャイル開発手法をもとにした独自のタスク管理手法を用いていることである。
バックログを導入することにより、タスク管理が容易となった。
また、開発フェーズに移行した際も、アジャイル開発手法を導入しやすいという利点がある。
2つ目は、プロジェクトリーダーを決めていないことである。
これにより、メンバー全員が責任感を持ち、活動しやすい環境が実現されている。

サービスの提案をするにあたり、フィールドワークを実施し、街に潜む課題・需要を見つけ出した。
さらに、「函館らしさ」「ビーコンである理由」「必要性」「新規性」「不変的な魅力(継続性)」の観点から、複数のアイデアを創出し、9個のアイデアに絞り込んだ。
後期では、まず、これらのアイデアをさらにブラッシュアップし、3つまたは4つのアイデアに絞り込む。
その後、開発を行う際に必要となるサービスの詳細設計や、要件定義の作成を行う。
それらをもとに、サービスの実装と実証実験を行う。
このプロセスを繰り返し行うことで、よりよいサービスを作る。

% 和文キーワード
\begin{jkeyword}
ビーコンIoT, 函館, 観光資源, 基盤産業, タスク管理手法, 街に潜む課題・需要
\end{jkeyword}
\bunseki{未来太郎}
\end{jabstract}

%英語の概要
\begin{eabstract} Abstract in English. 
% 英文キーワード
\begin{ekeyword}
Keyrods1, Keyword2, Keyword3, Keyword4, Keyword5
\end{ekeyword}
\bunseki{函館花子}
\end{eabstract}

\tableofcontents% 目次


\mainmatter% 本文のはじまり

\chapter{本プロジェクトの活動と目的}

\section{背景}


\section{目的}


\section{ビーコンについて}


\section{プロジェクトの進め方の決定}
%アジャイル的タスク管理手法について述べる



\chapter{提供サービスの考案プロセス}
%例年は「グループ課題設定までのプロセス」となっている.

\section{プロセス概要}


\section{フィールドワーク}

\subsection{事前調査}

\subsection{フィールドワークに関するレクチャー}

\subsection{実地調査}

\subsection{振り返り}


\section{サービスの考案}

\subsection{BSKJ法(ブレーンストーミングKJ法)}

\subsection{ハッカソン方式}

\subsection{OST(オープンスペーステクノロジー)}

\subsection{中間発表評価アンケートを用いたアイデアへの意見}

\subsection{アイデアのブラッシュアップ}

\subsection{テーマの決定}


\section{その他}

\subsection{ロゴ作成}

\subsection{ビーコンについての事前調査}

\subsection{夏休み勉強会の実施計画}

\subsection{昨年度のサービスの説明}



\chapter{提案するサービスについて}
%絞り込んだ仮アイデアの一覧をのせようかなぁ

\section{概要}


\section{サービス1}

\subsection{背景}

\subsection{目的}

\subsection{etc}

\section{サービス2}

\subsection{背景}

\subsection{目的}

\subsection{etc}


\chapter{中間発表}

\section{発表形式}

\section{レビュー内容}

\subsection{発表形式の評価と反省}

\subsection{発表内容の評価と反省}


\chapter{今後の予定}

\section{技術取得}

\section{実装}

\section{実地試験}


\chapter{学び}

\section{TBA:}


% 以降、付録(付属資料)であることを示す
\begin{appendix}

\chapter{中間発表会で使用したプロジェクト概要のポスター}

%付録の終わり
\end{appendix}


%\backmatter

% 参考文献
\begin{thebibliography}{9}
 \bibitem {ラベル} 著者名. 書籍名. 出版社,  年号.
 \bibitem {A2} ほげほげお. うんたらかんたら,  2003.
\end{thebibliography}

\end{document}
