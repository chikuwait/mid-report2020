\chapter{中間発表}

\section{発表形式}
 中間発表はオンラインでの開催であった。始めに、本プロジェクトの概要や仮アイデアの大まかな説明を行うため、10分間の動画とポスターを使用して発表した。その後、Zoomを使用して15分間の質疑応答を6回行った。
\bunseki{hoge}

\section{レビュー内容}
\subsection{発表形式の評価と反省}
 発表技術に関して、高評価な意見として
\begin{itemize}
    \item 「スライドがわかりやすくていいと思った。」
    \item 「発表動画がわかりやすくて、どのようなことを行ってきたかよく理解できた。」
    \item 「動画のクオリティがすごい。」
\end{itemize}
などが得られた。以上から聞き手にわかりやすく伝えることができたと考えられるため、発表技術に問題はなかったと言える。低評価な意見としては、
\begin{itemize}
    \item 「音声が聞き取りにくく内容を理解し難かった。」
    \item 「個人的な好みとしては生身の発表者の方がよいのではと思う。」
    \item 「音声がとにかく聴きづらかった。」
\end{itemize}
などが得られた。アンケートは46件の回答が得られ、平均評価は10点中、7.8点であった。以上から、合成音声になじみが無い人にとっては聞き取りずらい動画発表であったことがわかった。
\bunseki{hoge}

\subsection{発表内容の評価と反省}
 発表内容に関しては、高評価な意見として
\begin{{itemize}
    \item 「プロジェクトを進める手法については非常に効率的で、説明もわかりやすかった。」
    \item 「プロジェクトの目的や計画が明確に示されていた。」
    \item 「今後に希望が持てる内容であった。」
\end{itemize}
などが得られた。低評価な意見としては、
\begin{itemize}
    \item 「何をしたいのかよくわかりませんでした。」
    \item 「専門用語について説明があると分かり易いと思った。」
\end{itemize}
などが得られた。アンケートは46件の回答が得られ、プロジェクトの計画および、目標設定についての平均評価はどちらも6点中、4.7点であった。多くが高評価であり、今後に期待できるとの回答であったため、良かった点を活かしていきたい。